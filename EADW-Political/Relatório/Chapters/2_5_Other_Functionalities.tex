\subsection{Outras Funcionalidades}
\hspace{15pt}Realizámos a recolha de nomes de cada um dos ministros do actual governo e marcámos as entidades como membros do governo. Além disso, fizemos parsing ao site da assembleia da republica e recolhemos todos os nomes de deputados e respectivos partidos e associamos às respectivas entidades.\\
Com base nesta contextualização das entidades, realizámos a análises entre partidos e da avaliação do governo.

\subsubsection{Análise Partidária e Governamental}%Dario
Obtida a lista que associa uma entidade a um partido político e/ou ao governo, 
podemos somar todas as opiniões e verificar qual a opinião que a imprensa 
pública sobre cada partido e sobre o governo. \\
Os resultados deste método são bastante bons. É possível observar que os 
partidos politicos com maior relevância (PS e PSD) têm uma opinião bastante 
oscilatória enquanto que partidos mais pequenos são pouco referênciados. Isto 
denota a inclinação partidária dos média bem como a evolução da opinião que é 
transmitida.

\subsubsection{Caracterização das entidades}
\hspace{15pt}Recolhemos em cada frase os adjectivos que caracterizam as entidades dessa frase e associámos estes adjectivos à entidade. Deste modo, não só sabemos a opinião como também quais as características mais faladas a cada uma das 
entidades. A recolha destes adjectivos permite caracterizar com mais detalhe 
como é que a impressa avalia uma determinada entidade.

\subsubsection{Evolução da opinião}
\hspace{15pt}Para cada entidade, podemos consultar o número de noticias positivas e o nº de noticias negativas e deste modo saber como é que a popularidade da entidade evoluiu ao longo do tempo. 

\subsubsection{Interface de Consulta}
\hspace{15pt}A percepção sobre os dados recolhidos aumenta quando analisádos gráficamente. Para tal, criámos uma interface web de pesquisa. Esta interface foi criada no servidor Python Bottle \cite{bottle}. 
e da biblioteca de gráficos javascript nvd3. Requer algum trabalho mas permite 
converter o que em números é difícil analisar em gráficos que sumarizam os 
valores numéricos e permitem inferir resultados globais como a evolução ao longo 
do tempo.

\subsubsection{Mecanismo de caching}
\hspace{15pt}Durante a execução dos testes encontramos um \textit{bottleneck} no caso da ligação de Internet. O problema consistia no tempo que cada pagina demorava a descarregar. Para tal criamos mecanismo de caching de paginas e de descarregamento em paralelo. A ideia baseia-se em descarregar todas as paginas com um processo em multithread para um cache local. Isso diminuiu em um terço o tempo nessásario para processar a nossa base de dados. Por outro lado como todas as paginas permaneciam em cache as seguintes leituras seriam locais. O nome dos ficheiros é dado utilizando o url original e usando a função de resumo SHA-1 o que assegura a unicidade dos ficheiros e evita mecanismos complexos para a atribuição dos nomes num sistema de cache.

\subsubsection{Garbage Collector de Entidades}
\hspace{15pt}Como referido em \textbf{2.3} a nossa solução utiliza uma base de dados de nomes que provem de varias fontes, por outro lado o nosso sistema contém um mecanismo que tenta determinar entidades novas. Este tipo de abordagem gera muitos falsos positivos. No Nosso caso queremos nos focar nas entidades políticas que podem nem sempre ser as mais predominantes num artigo. Para resolver este problema criamos um mecanismo de treino. Este consiste em processar todos os artigos obtidos por forma a encontrar todas as entidades. Para cada entidade encontrada vamos manter um contador. No final para as cem entidades mais referenciadas vamos perguntar ao utilizador quais destas são realmente Entidades Politicas. As decisões tomadas pelo utilizador são guardadas em duas listas uma \textbf{'White List'} e uma \textbf{'Black List'}. A Black List é utilizada em conjunto com 'Stop Words' para filtrar o conteúdo, desta forma estas palavras não são classificadas e não contam para as estatísticas das entidades. Por outro lado a White List é utilizada como filtro para o \textit{triningSet} desta forma utilizador ira sempre receber novas palavras para classificar.\\
Esta técnica permitiu retirar muitas palavras que não representavam entidades e os resultados podiam ser visto nas novas pesquisas efectuadas melhorando a precisão das nossas pesquisas. \\